\begin{chapter}[beamerthemesrc/assets/background_negative]{}{Les méthodes agiles}
    \begin{itemize}
          \item Organisation de l'équipe
          \item Objectifs des releases
          \item Planification des sprints
    \end{itemize}
\end{chapter}

\begin{frame}{Organisation de l'équipe}
    \begin{block}{Scrum Master}
        \begin{itemize}
            \item Deux méthodes de sélection
        \end{itemize}
    \end{block}
    \begin{block}{Testeur}
        \begin{itemize}
            \item Une personne choisi
        \end{itemize}
    \end{block}
    \begin{block}{Equipe de développement}
        \begin{itemize}
            \item Une équipe de développeur diviser en fonctions des modules
        \end{itemize}
    \end{block}
\end{frame}

\begin{frame}{Objectifs des releases}
    \begin{block}{Release 1}
        \begin{itemize}
            \item Interpréter sur les opérateurs arithmétique
            \item Compiler sur les opérateurs arithmétique
        \end{itemize}
    \end{block}

    \begin{block}{Release 2}
        \begin{itemize}
            \item Contrôler les types
            \item Interpréter \& compiler les tableaux, les variables
            \item Avoir un IDE
        \end{itemize}
    \end{block}
\end{frame}

\begin{frame}{Planification des sprints}
    \begin{block}{}
        \centering
        Développement incrémental
    \end{block}

    \begin{block}{Release 1}
        \begin{itemize}
            \item Implémentation de l'analyseur syntaxique
            \item Implémentation de l'interpréteur MiniJaja \& de la pile
            \item Implémentation du compilateur \& du contrôleur de type
        \end{itemize}
    \end{block}

    \begin{block}{Release 2}
        \begin{itemize}
            \item Implémentation du tas
            \item Implémentation de l'interpréteur JajaCode
            \item Intégration des modules à l'IDE
        \end{itemize}
    \end{block}
\end{frame}

% \begin{frame}{CI/CD}
%     \begin{block}{GitLab}
%         \begin{itemize}
%             \item Utilisation de jobs avec GitLab
%             \item Développement sur des branches séparées (une branche par développeur)
%             \item Organisation des revues de code pour les Merge Request
%             \item Déployement des artefacts et du code sur Nexus \& SonarQube
%         \end{itemize}
%     \end{block}

%     \begin{block}{SonarQube \& Nexus}
%         \begin{itemize}
%             \item Utilisation de SonarQube pour l'analyse de code
%             \item Utilisation de Nexus pour le stockage des artefacts
%         \end{itemize}
%     \end{block}
% \end{frame}